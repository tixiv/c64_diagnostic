% !TEX TS-program = pdflatex
% !TEX encoding = UTF-8 Unicode

% This is a simple template for a LaTeX document using the "article" class.
% See "book", "report", "letter" for other types of document.

\documentclass[11pt]{article} % use larger type; default would be 10pt

\usepackage[utf8]{inputenc} % set input encoding (not needed with XeLaTeX)

%%% Examples of Article customizations
% These packages are optional, depending whether you want the features they provide.
% See the LaTeX Companion or other references for full information.

%%% PAGE DIMENSIONS
\usepackage{geometry} % to change the page dimensions
\geometry{a4paper} % or letterpaper (US) or a5paper or....
% \geometry{margin=2in} % for example, change the margins to 2 inches all round
% \geometry{landscape} % set up the page for landscape
%   read geometry.pdf for detailed page layout information

\usepackage{graphicx} % support the \includegraphics command and options

\usepackage[parfill]{parskip} % Activate to begin paragraphs with an empty line rather than an indent

%%% PACKAGES
\usepackage{booktabs} % for much better looking tables
\usepackage{array} % for better arrays (eg matrices) in maths
\usepackage{paralist} % very flexible & customisable lists (eg. enumerate/itemize, etc.)
\usepackage{verbatim} % adds environment for commenting out blocks of text & for better verbatim
\usepackage{subfig} % make it possible to include more than one captioned figure/table in a single float

\usepackage{german}   

% These packages are all incorporated in the memoir class to one degree or another...

%%% HEADERS & FOOTERS
%%\usepackage{fancyhdr} % This should be set AFTER setting up the page geometry
%%\pagestyle{fancy} % options: empty , plain , fancy
%%\renewcommand{\headrulewidth}{0pt} % customise the layout...
%%\lhead{}\chead{}\rhead{}
%%\lfoot{}\cfoot{\thepage}\rfoot{}


%%% SECTION TITLE APPEARANCE
\usepackage{sectsty}
\allsectionsfont{\sffamily\mdseries\upshape} % (See the fntguide.pdf for font help)
% (This matches ConTeXt defaults)

%%% ToC (table of contents) APPEARANCE
\usepackage[nottoc,notlof,notlot]{tocbibind} % Put the bibliography in the ToC
\usepackage[titles,subfigure]{tocloft} % Alter the style of the Table of Contents
\renewcommand{\cftsecfont}{\rmfamily\mdseries\upshape}
\renewcommand{\cftsecpagefont}{\rmfamily\mdseries\upshape} % No bold!

\renewcommand{\figurename}{Abb.}

%%% END Article customizations

%%% The "real" document content comes below...

\title{Tixivs Ultimate Dead Test Cartridge}
\author{}
\date{} % Activate to display a given date or no date (if empty),
         % otherwise the current date is printed 

\begin{document}
\maketitle


\tableofcontents

\newpage

\section{Introduction}

Testing dead C64s (black screen of death) can be a real challenge, because almost every of it's components could be causing the problem.

First steps:
\begin{itemize}
    \item check if the Voltages are o.k. Test to GND with a multimeter or scope:
	\begin{itemize}
		\item CPU Pin 4: 5VDC from Powersupply brick (check power supply, power switch and the soldering on the filter coil next to the switch if voltage is missing)
		\item VIC Pin 40 or 7805 Pin 3: 5VDC(VIC, oscillator) from 7805 (which is powered from 9VAC, check fuse if missing)
		\item 7812 Pin 3: 12VDC(VIC,SID) from 7812 (which is also powered from 9VAC, check fuse if missing)
	\end{itemize}
    \item check the reset
	\begin{itemize}
		\item CPU Pin 40: reset should be low first, then go high (check U8(7406) and U20(NE556) if missing
	\end{itemize}
    \item check the clocks
	\begin{itemize}
		\item CPU Pin 1: Phi0 from VIC (0.98MHz)
		\item CPU Pin 39: Phi2 Generated by CPU (also 0.98MHz but different phase)
		\item VIC Pin 21: Phi-color from oscillator (17.73MHz)
		\item VIC Pin 22: Dot Clock from PLL (7.88MHz = (Phi-color / 9) * 4)
	\end{itemize}	
\end{itemize}


Test Ablauf:
\begin{itemize}
	\item C64 startet im Ultimax Modus: 4k RAM \$0000-\$0FFF, Cartridge ROM at \$8000-\$9FFF and \$E000-\$FFFF
	\item VIC und CIAs werden initialisiert, screen=\$400-\$7FF, char=\$800-\$BFF
	\item Charset wir kopiert und Einschaltmeldung angezeigt
	\item SID test: auf jedem Kanal wird ein Ton (440Hz) abgespielt (deng deng deng)
	\item RAM Address test: Es wird die Addressierung des Speichers durch die CPU geprüft, Allerdings nur A0-A11(wegen Ultimax Modus). Bei erfolg hoher ton(ding) sonst fehlercode. Zeigt der Fehlercode schon bei diesem Test eine oder mehrere defekte Datenleitungen an, so funktioniert nichtmal die für disen test benutze Speicherzelle (\$FFF) zuverlässig.
	\item RAM test: (deng) Es werden alle Speicherzellen mit dem gleichen testbyte gefüllt, und nach einer Verzögerung wieder ausgelesen. Stimmen die testbytes überein gibt es einen hohe ton pro zeichen (ding * 21), sonst einen fehlercode
	\item C64 wechselt in den 8k Cartridge modus, indem das Flipflop auf der Cartridge durch einen Zugriff auf die IO2-Basisaddresse (\$DF00) umgeschaltet wird. Leuchtdiode auf Cart geht aus. Cartridge nurnoch bei \$8000-\$9FFF., voller Zugriff auf alllen RAM und alle ROMs
	\item SID test: auf jedem Kanal wird ein Ton (440Hz) abgespielt (deng deng deng)
	\item RAM Address test: Es wird die Addressierung des Speichers durch die CPU geprüft, nun für alle Addressleitungen. Bei erfolg hoher ton(ding) sonst fehlercode.
	\item VIC Address test: Der RAM wird so gefüllt, dass je nach defekter Video Addressleitung (VA15, VA14, VA6, VA7)Addressleitung der Name der defekten Leitung durch den Pattern im Colorram rot erscheint VA6 und VA7 erscheinen normalerweise blau, VA14 und VA15 erscheinen nicht. Leitungen die in Ordnung sind erscheinen blau.

\end{itemize}
\begin{table}[h!]
  \begin{center}
    \caption{Blink codes for defetive RAM}
    \label{tab:table1}
    \begin{tabular}{c|l|l|l|l}
      \textbf{Number of blinks} & \textbf{Problem on line} & \textbf{Normal} & \textbf{250466 (Only 2 RAMs)} & \textbf{250469 (short board)}\\
      \hline
      10 & A0 & U25 (74LS257) & U25 (74LS257) & U8 (PLA)\\
      11 & A1 & U25 (74LS257) & U25 (74LS257) & U8 (PLA)\\
      12 & A2 & U25 (74LS257) & U25 (74LS257) & U8 (PLA)\\
      13 & A3 & U25 (74LS257) & U25 (74LS257) & U8 (PLA)\\
      14 & A4 & U13 (74LS257) & U13 (74LS257) & U8 (PLA)\\
      15 & A5 & U13 (74LS257) & U13 (74LS257) & U8 (PLA)\\
      16 & A6 & U13 (74LS257) & U13 (74LS257) & U8 (PLA)\\
      17 & A7 & U13 (74LS257) & U13 (74LS257) & U8 (PLA)\\
      18 & A8 & U25 (74LS257) & U25 (74LS257) & U8 (PLA)\\
      19 & A9 & U25 (74LS257) & U25 (74LS257) & U8 (PLA)\\
      20 & A10 & U25 (74LS257) & U25 (74LS257) & U8 (PLA)\\
      21 & A11 & U25 (74LS257) & U25 (74LS257) & U8 (PLA)\\
      22 & A12 & U13 (74LS257) & U13 (74LS257) & U8 (PLA)\\
      23 & A13 & U13 (74LS257) & U13 (74LS257) & U8 (PLA)\\
      24 & A14 & U13 (74LS257) & U13 (74LS257) & U8 (PLA)\\
      25 & A15 & U13 (74LS257) & U13 (74LS257) & U8 (PLA)\\
      1 & D0 & U21 (RAM 4164) & U10 (RAM 50464) & U10 (RAM 41464)\\
      2 & D1 & U9 (RAM 4164) & U10 (RAM 50464) & U10 (RAM 41464) \\
      3 & D2 & U22 (RAM 4164) & U10 (RAM 50464) & U10 (RAM 41464) \\
      4 & D3 & U10 (RAM 4164) & U10 (RAM 50464) & U10 (RAM 41464) \\
      5 & D4 & U23 (RAM 4164) & U9 (RAM 50464) & U11 (RAM 41464) \\
      6 & D5 & U11 (RAM 4164) & U9 (RAM 50464) & U11 (RAM 41464) \\
      7 & D6 & U24 (RAM 4164) & U9 (RAM 50464) & U11 (RAM 41464) \\
      8 & D7 & U12 (RAM 4164) & U9 (RAM 50464) & U11 (RAM 41464) \\
    \end{tabular}
  \end{center}
\end{table}


\end{document}
